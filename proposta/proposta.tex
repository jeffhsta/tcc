\documentclass[11pt,a4paper]{article}

\usepackage{iccv}
\usepackage{times}
\usepackage{epsfig}
\usepackage{graphicx}
\usepackage{amsmath}
\usepackage{amssymb}
\usepackage[brazil]{babel}
\usepackage[OT1]{fontenc}
\usepackage[utf8]{inputenc}
\usepackage[a4paper,
  hmargin={1.5cm,1.5cm},
  vmargin={2cm,2.5cm},
footskip=5mm]{geometry}

\usepackage[pagebackref=true,breaklinks=true,letterpaper=false,colorlinks,bookmarks=false]{hyperref}

\begin{document}

\title{Automação de ambientes utilizando Docker\\ \smallskip
\small{PROPOSTA DE TRABALHO DE CONCLUSÃO DE CURSO}}

\author{ \bf Jefferson Heckler Stachelski\\
  \tt jeffhsta@riseup.net \\
  Curso de Sistemas de Informação \\
  Centro Universitário Ritter Dos Reis - UNIRITTER
  \and
  \bf Guilherme Lacerda\\
  Professor Orientador\\
}

\maketitle
\thispagestyle{empty}

\section{Relevância do trabalho} \label{sec:intro}

Muitos desenvolvedores de software trabalham usando o seu computador como ambiente principal para rodar e testar
o software durante o processo de desenvolvimento do software. Assim esse ambiente constuma ser diferente dos ambientes
dos servidores onde o software irá ser instalado, como os ambientes de desenvolvimento, homologação e produção.
Com isso em alguns casos ocorrem problemas de imcompatibilidade do software entre os ambientes, geralmente causando
por versões de sistemas operacionais ou ferramentas diferentes. O uso de containers por desenvolvedores é de apenas
17\%, 56\% estão considerando a adoção da prática e 27\% não utilizam e não possuem planos de adoção\cite{DZone_CD_guide_v3}.

A \textit{automação} e \textit{virtualização} de processos de \textit{configuração de ambientes} e de \textit{deploy} tem
se tornado uma prática cada vez mais comum\cite{DZone_CD_guide_v3}. Essa prática acaba minimizando as falhas de imcompatibilidade do software
nos ambientes no qual esse é implantado\cite{Fowler_continuos_integration}. A \textit{virtualização} é uma prática
adotada muitas vezes no processo de desenvolvimento fazendo com que o software seja desenvolvido e testado sobre
esse ambiente virtualizado, afim de que mantenha-se a compatibilidade com o ambiente de produção. Com a
\textit{virtualização} no processo de desenvolvimento acaba-se reduzindo alguns riscos e até mesmo facilitando
quando há alguma mudança no ambiente de produção, por exemplo uma atualização de uma biblioteca ou qualquer
outra dependencia do software\cite{Fowler_continuos_integration}.

Porém o consumo de recursos do computador tem se mostrado muito elevado para possuir esse ambiente virtualizado,
além de o processo de configuração desse ambiente ser muitas vezes demorado, de acordo com a complexidade e
o número de dependências que esse possui. Como alternativa a \textit{virtualização} temos \textit{containers}.
\textit{Containers} é o mecanismo utilizado pela ferramente chamada \textit{Docker}, que ao invés de virtualizar
um sistema operacional, esse mecanismo utiliza muitos dos componentes ja existentes no computador, e executa
outros componentes a fim de isolar os processos executados dentro do container\cite{TW_docker_for_builds},
fazendo assim essa uma alternativa, mais rápida e que consume menos recursos do computador\cite{DZone_CD_guide_v3}.
Um bom caso de exemplo seria um ambiente onde estão rodando \textit{microservices}, cujos são compostoso por
outros serviços externos, no qual cada \textit{microservice} deve ser resiliente, flexivel, mínimo e completo
\cite{Bugwadia_arch_constrains}.

No contexto abordado acima são raros os trabalhos que apontam a vantagem do uso do \textit{Docker} como uma solução
alternativa a \textit{virtualização}\cite{Jafari_infra_as_a_code}. Quando trata-se desse tópico geralmente são
discutidos ferramentas de automação que criam uma maquina virtual e a configuram, quando que esse processo
pode possuir menos tarefas repetitivas e um processo de criação e configuração de um ambiente de forma mais rapido e menos complexa.

\section{Objetivos}\label{sec:objetivos}

\subsection{Objetivo geral}

Este trabalho propõem um estudo de caso da aplicação do Docker em aplicação corporativa analisando os aspectos
de performance, segurança, isolamento, tempo de liberação compado maquinas virtuais.

\subsection{Objetivos específicos}

Entre os objetivos específicos deste trabalho pode-se destacar:

\begin{itemize}
  \item Avaliar a complexidade, recursos consumidos e tempo que leva-se para criar e configurar um ambiente
    utilizando uma maquina virtual.
  \item Analizar qual o esforço ao atualizar um componente que é uma dependencia do ambiente.
  \item Avaliar a complexidade, recursos consumidos e tempo que leva-se para criar e configurar um ambiente
    utilizando maquina virtual.
    utilizando containers Docker.
  \item Analizar qual o esforço ao atualizar um componente que é uma dependencia do ambiente utilizando Docker.
  \item Prover um estudo inicial através de um questionário sobre a diferença entre automatizar
    ambientes virtualizados e ambientes sobre containers Docker.
\end{itemize}

\section{Solução proposta}

Este trabalho visa avaliar os beneficios de utilizar containers Docker como método de automatizar os ambientes
ao invés do uso de maquinas virtuais. Especificamente, busca-se avaliar a redução de complexidade, desacoplamento dos
componentes e a queda de uso de recursos do computador ao utilizar a abordagem de containers como ambientes para instalação
do software.

Os experimentos serão conduzidos através de um formulario com perguntas relacionadas a adoção de containers por um time
de desenvolvimento de software, cujo adotou a utilização de Docker com a finalidade de melhorar o processo de automação
de ambientes de desenvolvimento, testes e produção. O questionário será aplicado a um time de pelo menos cinco desenvolvedores,
as perguntas do forumalário estarão relacionadas a nivel de complexidade, tempo de resposta e recursos utilizados do computador
antes e após a adoção do Docker no projeto.

\section{Cronograma de desenvolvimento}\label{sec:cronograma}

\subsection{Trabalho de conclusão de curso I}

A Tabela \ref{tab:cronograma1} apresenta o cronograma de desenvolvimento do trabalho conforme a numeração das atividades abaixo:
\begin{enumerate}
  \item Levantamento das variaveis que diferenciam utilização de maquinas virtuais e containers;
  \item Elaboração do formulário com as perguntas relavantes com base no item acima;
  \item Validação do formulário elaborado com o orientador e desenvolvedores a fim de coletar feedbacks;
  \item Elaboração da versão final do formulário;
  \item Redação do artigo;
  \item Submissão do artigo para a banca;
  \item Defesa do trabalho.
\end{enumerate}

\begin{table}[h]
  \begin{center}
    \caption{Cronograma de atividades para o trabalho de conclusão I. \label{tab:cronograma1}}
    \begin{tabular}{|c|c|c|c|c|c|c|}
      \hline
      \bf Atividade & \bf 3/16 & \bf 4/16 & \bf 5/16 & \bf 6/16 & \bf 7/16 \\  \hline \hline
      1 & x & x &   &   &   \\ \hline
      2 &   & x & x &   &   \\ \hline
      3 &   &   & x & x &   \\ \hline
      4 &   &   & x & x &   \\ \hline
      5 &   &   &   & x & x \\ \hline
      6 &   &   &   & x & x \\ \hline
      7 &   &   &   &   & x \\ \hline
    \end{tabular}
  \end{center}
\end{table}

\subsection{Trabalho de conclusão de curso II}

A Tabela \ref{tab:cronograma2} apresenta o cronograma de desenvolvimento do trabalho conforme a numeração das atividades abaixo:
\begin{enumerate}
  \item Definição do \textit{setup} de experimentos e métricas de validação com base nos trabalhos relacionados;
  \item Implementação dos experimentos;
  \item Análise de resultados preliminares e ajuste de melhores configurações de parâmetros para os algoritmos de mineração de texto empregados;
  \item Análise comparativa dos resultados com relação à trabalhos relacionados;
  \item Redação do artigo;
  \item Seminário de andamento;
  \item Submissão do artigo para a banca;
  \item Defesa do trabalho.
\end{enumerate}

\begin{table}[h]
  \begin{center}
    \caption{Cronograma de atividades para o trabalho de conclusão II. \label{tab:cronograma2}}
    \begin{tabular}{|c|c|c|c|c|c|c|}
      \hline
      \bf Atividade & \bf 2/13 & \bf 3/13 & \bf 4/13 & \bf 5/13 & \bf 6/13 & \bf 7/13  \\  \hline \hline
      1 & x &   &   &   &   &   \\ \hline
      2 & x & x & x & x &   &   \\ \hline
      3 &   & x & x & x &   &   \\ \hline
      4 &   &   & x & x &   &   \\ \hline
      5 &   &   & x & x & x &   \\ \hline
      6 &   &   &   & x  &   &   \\ \hline
      7 &   &   &   &   & x &   \\ \hline
      8 &   &   &   &   &   & x \\ \hline
    \end{tabular}
  \end{center}
\end{table}

\renewcommand\refname{Referências}
{\small
  \bibliographystyle{ieee}
  \bibliography{referencias}
}

\end{document}
