\documentclass[11pt,a4paper]{article}

\usepackage{iccv}
\usepackage{times}
\usepackage{epsfig}
\usepackage{graphicx}
\usepackage{amsmath}
\usepackage{amssymb}
\usepackage[brazil]{babel}
\usepackage[OT1]{fontenc}
\usepackage[utf8]{inputenc}
\usepackage[a4paper,
  hmargin={1.5cm,1.5cm},
  vmargin={2cm,2.5cm},
footskip=5mm]{geometry}

\usepackage[pagebackref=true,breaklinks=true,letterpaper=false,colorlinks,bookmarks=false]{hyperref}

\begin{document}

\title{Automação de ambientes utilizando Docker\\ \smallskip
\small{PROPOSTA DE TRABALHO DE CONCLUSÃO DE CURSO}}

\author{ \bf Jefferson Heckler Stachelski\\
  \tt jeffhsta@riseup.net \\
  Curso de Sistemas de Informação \\
  Centro Universitário Ritter Dos Reis - UNIRITTER
  \and
  \bf Guilherme Lacerda\\
  Professor Orientador\\
}

\maketitle
\thispagestyle{empty}

\section{Relevância do trabalho} \label{sec:intro}

Muitos desenvolvedores de \textit{software} trabalham usando o seu computador como ambiente principal para rodar e testar
o \textit{software} durante o processo de desenvolvimento do \textit{software}. Assim esse ambiente constuma ser diferente dos ambientes
dos servidores onde o \textit{software} irá ser instalado, como os ambientes de desenvolvimento, homologação e produção.
Com isso em alguns casos ocorrem problemas de imcompatibilidade do \textit{software} entre os ambientes, geralmente causando
por versões de sistemas operacionais ou ferramentas diferentes. O uso de \textit{containers} por desenvolvedores é de apenas
17\%, 56\% estão considerando a adoção da prática e 27\% não utilizam e não possuem planos de adoção\cite{DZone_CD_guide_v3}.

A \textit{automação} e \textit{virtualização} de processos de \textit{configuração de ambientes} e de \textit{deploy} tem
se tornado uma prática cada vez mais comum\cite{DZone_CD_guide_v3}. Essa prática acaba minimizando as falhas de
imcompatibilidade do \textit{software} nos ambientes no qual esse é implantado\cite{Fowler_continuos_integration}.
A \textit{virtualização} é uma prática adotada muitas vezes no processo de desenvolvimento fazendo com que o
\textit{software} seja desenvolvido e testado sobre esse ambiente virtualizado, afim de que mantenha-se a compatibilidade
com o ambiente de produção, em muitos casos usa-se ferramentas de provisionamento do ambiente em maquinas virtuais
como por exemplo \textit{Cheff}, \textit{Puppet} entre outros\cite{Fowler_infra_as_code}. Com a \textit{virtualização}
no processo de desenvolvimento acaba-se reduzindo alguns riscos e até mesmo facilitando quando há alguma mudança
no ambiente de produção, por exemplo uma atualização de uma biblioteca ou qualquer outra dependencia do
\textit{software}\cite{Fowler_continuos_integration}.

Porém o consumo de recursos do computador tem se mostrado muito elevado para possuir esse ambiente virtualizado,
além de o processo de configuração desse ambiente ser muitas vezes demorado, de acordo com a complexidade e
o número de dependências que esse possui. Como alternativa a \textit{virtualização} temos \textit{containers}.
\textit{Containers} é o mecanismo utilizado pela ferramente chamada \textit{Docker}, que ao invés de virtualizar
um sistema operacional, esse mecanismo utiliza muitos dos componentes ja existentes no computador, e executa
outros componentes a fim de isolar os processos executados dentro do \textit{container}\cite{TW_docker_for_builds},
fazendo assim essa uma alternativa, mais rápida e que consume menos recursos do computador\cite{DZone_CD_guide_v3}.
Um bom caso de exemplo seria um ambiente onde estão rodando \textit{microservices}, cujos são compostoso por
outros serviços externos, no qual cada \textit{microservice} deve ser resiliente, flexivel, mínimo e completo
\cite{Bugwadia_arch_constrains}. \textit{Docker} torna-se util também em cenários onde aplicações dependem umas
das outras, mas po algum motivo elas executão em diferentes versões de linguagens, biblioteca ou outra dependencia,
podendo rodar isoladamente cada aplicação em diferentes \textit{containers}\cite{Merkel_Docker}.

No contexto abordado acima são raros os trabalhos que apontam a vantagem do uso do \textit{Docker} como uma solução
alternativa a \textit{virtualização}\cite{Jafari_infra_as_a_code}. Quando trata-se desse tópico geralmente são
discutidos ferramentas de automação que criam uma maquina virtual e a configuram, quando que esse processo
pode possuir menos tarefas repetitivas e um processo de criação e configuração de um ambiente de forma mais
rapido e menos complexa.

\section{Objetivos}\label{sec:objetivos}

\subsection{Objetivo geral}

Este trabalho propõem um estudo de caso da aplicação do \textit{Docker} em aplicação corporativa analisando os aspectos
de performance, segurança, isolamento, tempo de liberação compado maquinas virtuais.

\subsection{Objetivos específicos}

Entre os objetivos específicos deste trabalho pode-se destacar:

\begin{itemize}
  \item Estudar as ferramentas de virtualização de \textit{software}, como por exemplo,
    VirtualBox, VWare, Hypervision entre outros. Conhecer as ferramentas existentes no mercado e as
    mas utilizadas pelas empresas.
  \item Estudar o \textit{Docker} a nivel prático, entender como funciona a ferramenta e quais são os
    recursos e funcionalidades disponiveis.
  \item Criar uma aplicação web simples, contendo poucas páginas, mas com dependencia de pelo menos duas
    aplicações externas.
  \item Automatizar a criação do ambiente do projeto citado acima utilizando uma ferramenta de virtualização
    previamente estudada.
  \item Automatizar a criação do ambiente como citado no item acima, porém utilizando o \textit{Docker}. Possuindo
    assim um cenário onde simula-se a migração de um ambiente virtualizado para utilização de \textit{containers}
  \item Prover um estudo inicial através de um questionário sobre a diferença entre automatizar
    ambientes com ferramentas de virtualização e ambientes com \textit{Docker}. Esse questionário será aplicado
    a desenvolvedores com algum nivel de expericência em \textit{Docker} a fim de prover obter uma visão geral
    do \textit{Docker} pelos profissionais da area de desenvolvimento de \textit{software}.
\end{itemize}

\section{Solução proposta}

Este trabalho visa avaliar os beneficios de utilizar \textit{containers} \textit{Docker} como método de automatizar
os ambientes ao invés do uso de maquinas virtuais. Esse trabalho difere-se de \textit{Performance comparison
analysis of linux container and virtual machine for building cloud}\cite{Performance_container_vm}, pois expecificamente
busca-se avaliar a redução de complexidade, desacoplamento dos componentes e a queda de uso de recursos do computador
ao utilizar a abordagem de \textit{containers} como base de provisionamento do ambiente de uma aplicação corporativa.

Os experimentos serão conduzidos através de um formulario com perguntas relacionadas a adoção de \textit{containers} por um time
de desenvolvimento de \textit{software}, cujo adotou a utilização de \textit{Docker} com a finalidade de melhorar o processo de automação
de ambientes de desenvolvimento, testes e produção. O questionário será aplicado a um time de pelo menos cinco desenvolvedores,
as perguntas do forumalário estarão relacionadas a nivel de complexidade, tempo de resposta e recursos utilizados do computador
antes e após a adoção do \textit{Docker} no projeto.

\section{Cronograma de desenvolvimento}\label{sec:cronograma}

\subsection{Trabalho de conclusão de curso I}

A Tabela \ref{tab:cronograma1} apresenta o cronograma de desenvolvimento do trabalho conforme a numeração das atividades abaixo:
\begin{enumerate}
  \item Levantamento das variaveis que diferenciam utilização de maquinas virtuais e \textit{containers};
  \item Elaboração do formulário com as perguntas relavantes com base no item acima;
  \item Validação do formulário elaborado com o orientador e desenvolvedores a fim de coletar feedbacks;
  \item Elaboração da versão final do formulário;
  \item Redação do artigo;
  \item Submissão do artigo para a banca;
  \item Defesa do trabalho.
\end{enumerate}

\begin{table}[h]
  \begin{center}
    \caption{Cronograma de atividades para o trabalho de conclusão I. \label{tab:cronograma1}}
    \begin{tabular}{|c|c|c|c|c|c|c|}
      \hline
      \bf Atividade & \bf 3/16 & \bf 4/16 & \bf 5/16 & \bf 6/16 & \bf 7/16 \\  \hline \hline
      1 & x & x &   &   &   \\ \hline
      2 &   & x & x &   &   \\ \hline
      3 &   &   & x & x &   \\ \hline
      4 &   &   & x & x &   \\ \hline
      5 &   &   &   & x & x \\ \hline
      6 &   &   &   & x & x \\ \hline
      7 &   &   &   &   & x \\ \hline
    \end{tabular}
  \end{center}
\end{table}

\subsection{Trabalho de conclusão de curso II}

A Tabela \ref{tab:cronograma2} apresenta o cronograma de desenvolvimento do trabalho conforme a numeração das atividades abaixo:
\begin{enumerate}
  \item Definição do \textit{setup} de experimentos e métricas de validação com base nos trabalhos relacionados;
  \item Implementação dos experimentos;
  \item Análise de resultados preliminares e ajuste de melhores configurações de parâmetros para os algoritmos de mineração de texto empregados;
  \item Análise comparativa dos resultados com relação à trabalhos relacionados;
  \item Redação do artigo;
  \item Seminário de andamento;
  \item Submissão do artigo para a banca;
  \item Defesa do trabalho.
\end{enumerate}

\begin{table}[h]
  \begin{center}
    \caption{Cronograma de atividades para o trabalho de conclusão II. \label{tab:cronograma2}}
    \begin{tabular}{|c|c|c|c|c|c|c|}
      \hline
      \bf Atividade & \bf 2/13 & \bf 3/13 & \bf 4/13 & \bf 5/13 & \bf 6/13 & \bf 7/13  \\  \hline \hline
      1 & x &   &   &   &   &   \\ \hline
      2 & x & x & x & x &   &   \\ \hline
      3 &   & x & x & x &   &   \\ \hline
      4 &   &   & x & x &   &   \\ \hline
      5 &   &   & x & x & x &   \\ \hline
      6 &   &   &   & x  &   &   \\ \hline
      7 &   &   &   &   & x &   \\ \hline
      8 &   &   &   &   &   & x \\ \hline
    \end{tabular}
  \end{center}
\end{table}

\renewcommand\refname{Referências}
{\small
  \bibliographystyle{ieee}
  \bibliography{referencias}
}

\end{document}
