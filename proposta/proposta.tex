\documentclass[11pt,a4paper]{article}

\usepackage{iccv}
\usepackage{times}
\usepackage{epsfig}
\usepackage{graphicx}
\usepackage{amsmath}
\usepackage{amssymb}
\usepackage[brazil]{babel}
\usepackage[OT1]{fontenc}
\usepackage[utf8]{inputenc}
\usepackage[a4paper,
hmargin={1.5cm,1.5cm},
vmargin={2cm,2.5cm},
footskip=5mm]{geometry}

\usepackage[pagebackref=true,breaklinks=true,letterpaper=false,colorlinks,bookmarks=false]{hyperref}

% \iccvfinalcopy % *** Retire o comentário desta linha para gerar a versão da biblioteca (após obter nota acima de 9 em TC2)

\begin{document}

\title{Caracterização Automática de Notícias no Contexto do Mercado de Ações Brasileiro \\ \smallskip
\small{ PROPOSTA DE TRABALHO DE CONCLUSÃO DE CURSO}}

\author{ \bf Jefferson Heckler Stachelski\\
    \tt jeffhsta@riseup.net \\
    Curso de Sistemas de Informação \\
    Centro Universitário Ritter Dos Reis - UNIRITTER
    \and
     \bf Guilherme Lacerda\\
    Professor Orientador\\
}

\maketitle
\thispagestyle{empty}


\section{Relevância do trabalho} \label{sec:intro}

Muitos investidores atuam no mercado de ações com o objetivo de realizar investimentos que possam ser rentáveis.
Neste sentido, uma previsão confiável do comportamento do mercado pode auxiliar os investidores em seus objetivos.
Contudo, esta não é uma tarefa fácil e ainda não há um método amplamente aceito para prever o movimento dos preços
das ações~\cite{Schumaker:2010}. Os desafios envolvem a compreensão da dinâmica do mercado, onde parâmetros mudam
constantemente e muitas vezes se quer estão (são) bem definidos(conhecidos). Por esta razão, a análise do mercado
de ações tem atraído o interesse da comunidade científica.

A \textit{análise técnica} e a \textit{análise fundamentalista} são duas das principais vertentes no contexto de
prever o comportamento do mercado financeiro \cite{Brum:acoes}. Basicamente, a análise técnica emprega dados históricos
de preços no processo de prever o comportamento futuro destes preços, enquanto que a análise fundamentalista baseia-se
em fatores econômicos que afetam o ramo de atividade das empresas, como indicadores econômicos e quebras de produção.
Neste contexto, eventos/notícias podem ter grande impacto no comportamento dos preços das ações no mercado e, por esta
razão, um dos pré-requisitos de um analista de mercado é estar a par de notícias associadas a ramos de atividades de
uma carteira de investimentos.

Com base no grande e crescente volume de informações que são gerados diariamente nos canais de notícias, se torna
impraticável para um analista monitorar uma grande quantidade de eventos relativos ao mercado de capitais, sobretudo
quando este processo deve ocorrer na velocidade das oscilações de preços dos ativos financeiros. Assim, para embasar
a tomada de decisão sobre títulos transacionados em bolsa de valores, se torna útil o processamento computacional
prévio destas informações de forma a entregar para o profissional especialista somente as informações relevantes ao contexto de interesse.

No contexto discutido acima são raros os trabalhos que envolvem o mercado nacional e sua suscetibilidade à notícias/eventos.
Quando o foco é o mercado de ações brasileiro, a literatura apresenta métodos que exploram o histórico de preços das ações
do ponto de vista de técnicas de análise de séries temporais~\cite{Edgard:Dissertacao}, porém não verificou-se trabalhos
que visam automatizar a busca por padrões de conteúdo em notícias (texto) como forma de entender o comportamento do mercado de ações.

\section{Objetivos}\label{sec:objetivos}

\subsection{Objetivo geral}

Este trabalho propõe um estudo sobre as potencialidades de métodos clássicos de mineração de texto focado na caracterização
automática do conteúdo de notícias que impactam no mercado de ações brasileiro.

\subsection{Objetivos específicos}

Entre os objetivos específicos deste trabalho pode-se destacar:

\begin{itemize}
  \item Avaliar a existência de padrões em notícias que versam sobre o mercado financeiro
  brasileiro em contextos de alta e baixa de preços.
  \item Investigar a influência de falsos positivos em dados de treinamento para a classificação
  supervisionada, uma vez que assume-se que todas as notícias que ocorrem em janelas temporais
  caracterizadas pela queda de preços são, de fato, notícias que levam a queda destes e vice-versa.
  \item Prover um estudo inicial sobre a bolsa brasileira no sentido de responder, de forma
  automatizada, perguntas como \textit{"O quão similar as notícias de hoje são em relação àquelas do passado?"},
  provendo suporte a decisão para analistas que buscam medir o impacto de determinados eventos sobre um conjunto de ações.
\end{itemize}

\section{Solução proposta}

Este trabalho visa avaliar o desempenho de uma abordagem \textit{bag-of-words} \cite{Manning:IR}
para propósitos de descoberta de padrões noticiosos e seus reflexos no mercado de ações brasileiro.
Especificamente, busca-se avaliar as potencialidades de uma abordagem clássica e difundida de mineração
de textos para a caracterização/classificação de textos noticiosos em contextos de alta e baixa de
preços no mercado de ações brasileiro. Neste sentido, avalia-se as potencialidades da abordagem proposta
para realizar previsões de curto e médio prazo sobre o comportamento do mercado acionário brasileiro.

Os experimentos serão conduzidos sobre uma base textual histórica que contém mais de 5000 notícias
sobre o mercado do petróleo e do aço no Brasil. Os textos provêm das principais fontes de notícias
brasileiras que publicam o seu conteúdo na internet, como Boletim Reuters, Correio Braziliense, Isto é Dinheiro,
O Estado de SP, O Globo, Folha de SP e Valor Econômico. As cotações dos ativos serão obtidas do site da Bovespa,
que mantém para consulta pública até dez anos de valores de abertura e fechamento de todos os papéis negociados pela entidade.


\section{Cronograma de desenvolvimento}\label{sec:cronograma}

\subsection{Trabalho de conclusão de curso I}

A Tabela \ref{tab:cronograma1} apresenta o cronograma de desenvolvimento do trabalho conforme a numeração das atividades abaixo:
\begin{enumerate}
  \item Levantamento de dados históricos (documentos e cotações);
  \item Pré-processamento dos dados históricos, como remoção de \textit{stopwords} e \textit{stemming} \cite{Weiss:textMining,orengo_stemming};
  \item Estudo sobre as principais técnicas de mineração de textos;
  \item Levantamento sobre abordagens de mineração de textos voltadas ao mercado de ações;
  \item Testes preliminares com bibliotecas de mineração de dados;
  \item Redação do artigo;
  \item Submissão do artigo para a banca;
  \item Defesa do trabalho.
\end{enumerate}

\begin{table}[h]
\begin{center}
  \caption{Cronograma de atividades para o trabalho de conclusão I. \label{tab:cronograma1}}
    \begin{tabular}{|c|c|c|c|c|c|c|}
      \hline
       \bf Atividade & \bf 7/12 & \bf 8/12 & \bf 9/12 & \bf 10/12 & \bf 11/12 & \bf 12/12  \\  \hline \hline
           1 & x & x &   &   &   &   \\ \hline
           2 &   & x & x & x &   &   \\ \hline
           3 & x & x & x &   &   &   \\ \hline
           4 & x & x & x & x &   &   \\ \hline
           5 &   &   &   & x & x &   \\ \hline
           6 &   &   & x & x & x &   \\ \hline
           7 &   &   &   &   & x &   \\ \hline
           8 &   &   &   &   &   & x \\ \hline
    \end{tabular}
    \end{center}
\end{table}


\subsection{Trabalho de conclusão de curso II}

A Tabela \ref{tab:cronograma2} apresenta o cronograma de desenvolvimento do trabalho conforme a numeração das atividades abaixo:
\begin{enumerate}
  \item Definição do \textit{setup} de experimentos e métricas de validação com base nos trabalhos relacionados;
  \item Implementação dos experimentos;
  \item Análise de resultados preliminares e ajuste de melhores configurações de parâmetros para os algoritmos de mineração de texto empregados;
  \item Análise comparativa dos resultados com relação à trabalhos relacionados;
  \item Redação do artigo;
  \item Seminário de andamento;
  \item Submissão do artigo para a banca;
  \item Defesa do trabalho.
\end{enumerate}

\begin{table}[h]
\begin{center}
  \caption{Cronograma de atividades para o trabalho de conclusão II. \label{tab:cronograma2}}
    \begin{tabular}{|c|c|c|c|c|c|c|}
      \hline
       \bf Atividade & \bf 2/13 & \bf 3/13 & \bf 4/13 & \bf 5/13 & \bf 6/13 & \bf 7/13  \\  \hline \hline
           1 & x &   &   &   &   &   \\ \hline
           2 & x & x & x & x &   &   \\ \hline
           3 &   & x & x & x &   &   \\ \hline
           4 &   &   & x & x &   &   \\ \hline
           5 &   &   & x & x & x &   \\ \hline
           6 &   &   &   & x  &   &   \\ \hline
           7 &   &   &   &   & x &   \\ \hline
           8 &   &   &   &   &   & x \\ \hline
    \end{tabular}
    \end{center}
\end{table}

\renewcommand\refname{Referências}
{\small
  \bibliographystyle{ieee}
  \bibliography{referencias}
}

\end{document}
