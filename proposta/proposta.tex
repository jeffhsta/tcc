\documentclass[11pt,a4paper]{article}

\usepackage{iccv}
\usepackage{times}
\usepackage{epsfig}
\usepackage{graphicx}
\usepackage{amsmath}
\usepackage{amssymb}
\usepackage[brazil]{babel}
\usepackage[OT1]{fontenc}
\usepackage[utf8]{inputenc}
\usepackage[a4paper,
hmargin={1.5cm,1.5cm},
vmargin={2cm,2.5cm},
footskip=5mm]{geometry}

\usepackage[pagebackref=true,breaklinks=true,letterpaper=false,colorlinks,bookmarks=false]{hyperref}

\begin{document}

\title{Automação de ambientes utilizando Docker\\ \smallskip
\small{PROPOSTA DE TRABALHO DE CONCLUSÃO DE CURSO}}

\author{ \bf Jefferson Heckler Stachelski\\
    \tt jeffhsta@riseup.net \\
    Curso de Sistemas de Informação \\
    Centro Universitário Ritter Dos Reis - UNIRITTER
    \and
     \bf Guilherme Lacerda\\
    Professor Orientador\\
}

\maketitle
\thispagestyle{empty}

\section{Relevância do trabalho} \label{sec:intro}

Muitos desenvolvedores de software trabalham usando o seu computador como ambiente principal para rodar e testar
o software durante o processo de desenvolvimento do software, assim esse ambiente constuma ser diferente dos ambientes
dos servidores onde o software irá ser instalado, como os ambientes de desenvolvimento, homologação e produção.
Com isso em alguns casos ocorrem problemas de imcompatibilidade do software entre os ambientes, geralmente causando
por versões de sistemas operacionais ou ferramentas diferentes.

A \textit{automação} e \textit{virtualização} de processos de \textit{configuração de ambientes} e de \textit{deploy} tem
se tornado uma prática cada vez mais comum. Essa prática acaba minimizando as falhas de imcompatibilidade do software
nos ambientes no qual esse é implantado\cite{Fowler_continuos_integration}. A \textit{virtualização} é uma prática adotada muitas vezes no processo
de desenvolvimento fazendo com que o software seja desenvolvido e testado sobre esse ambiente virtualizado, afim de
que mantenha-se a compatibilidade com o ambiente de produção. Com a \textit{virtualição} no processo de desenvolvimento
acaba-se reduzindo alguns riscos e até mesmo facilitando quando há alguma mudança no ambiente de produção, por exemplo
uma atualização de uma biblioteca ou qualquer outra dependencia do software\cite{Fowler_continuos_integration}.

Porém o consumo de recursos do computador tem se mostrado muito elevado para possuir esse ambiente virtualizado,
além de o processo de configuração desse ambiente ser muitas vezes demorado, de acordo com a complexidade e
o número de dependencies que esse possui. Como alternativa a \textit{virtualização} temos \textit{containers}.
\textit{Containers} é o mecanismo utilizado pela ferramente chamada \textit{Docker}, que ao invés de virtualizar
um sistema operacional, esse mecanismo utiliza muitos dos componentes ja existentes no computador, e executa
outros componentes a fim de isolar os processos executados dentro do container\cite{TW_docker_for_builds}, fazendo assim essa uma alternativa,
mais rápida e que consume menos recursos do computador.

No contexto abordado acima são raros os trabalhos que apontam a vantagem do uso do \textit{Docker} como uma solução
alternativa a \textit{virtualização}\cite{Jafari_infra_as_a_code}. Quando trata-se desse tópico geralmente são discutidos ferramentas de automação
que criam uma maquina virtual e a configuram, quando que esse processo pode possuir menos tarefas repetitivas e um
processo de criação e configuração de um ambiente de forma mais rapido e menos complexa.

\section{Objetivos}\label{sec:objetivos}

\subsection{Objetivo geral}

Este trabalho propõe um estudo sobre as potencialidades de métodos clássicos de mineração de texto focado na caracterização
automática do conteúdo de notícias que impactam no mercado de ações brasileiro.

\subsection{Objetivos específicos}

Entre os objetivos específicos deste trabalho pode-se destacar:

\begin{itemize}
  \item Avaliar a existência de padrões em notícias que versam sobre o mercado financeiro
    brasileiro em contextos de alta e baixa de preços.
  \item Investigar a influência de falsos positivos em dados de treinamento para a classificação
    supervisionada, uma vez que assume-se que todas as notícias que ocorrem em janelas temporais
    caracterizadas pela queda de preços são, de fato, notícias que levam a queda destes e vice-versa.
  \item Prover um estudo inicial sobre a bolsa brasileira no sentido de responder, de forma
    automatizada, perguntas como \textit{"O quão similar as notícias de hoje são em relação àquelas do passado?"},
    provendo suporte a decisão para analistas que buscam medir o impacto de determinados eventos sobre um conjunto de ações.
\end{itemize}

\section{Solução proposta}

Este trabalho visa avaliar o desempenho de uma abordagem \textit{bag-of-words} \cite{Manning:IR}
para propósitos de descoberta de padrões noticiosos e seus reflexos no mercado de ações brasileiro.
Especificamente, busca-se avaliar as potencialidades de uma abordagem clássica e difundida de mineração
de textos para a caracterização/classificação de textos noticiosos em contextos de alta e baixa de
preços no mercado de ações brasileiro. Neste sentido, avalia-se as potencialidades da abordagem proposta
para realizar previsões de curto e médio prazo sobre o comportamento do mercado acionário brasileiro.

Os experimentos serão conduzidos sobre uma base textual histórica que contém mais de 5000 notícias
sobre o mercado do petróleo e do aço no Brasil. Os textos provêm das principais fontes de notícias
brasileiras que publicam o seu conteúdo na internet, como Boletim Reuters, Correio Braziliense, Isto é Dinheiro,
O Estado de SP, O Globo, Folha de SP e Valor Econômico. As cotações dos ativos serão obtidas do site da Bovespa,
que mantém para consulta pública até dez anos de valores de abertura e fechamento de todos os papéis negociados pela entidade.


\section{Cronograma de desenvolvimento}\label{sec:cronograma}

\subsection{Trabalho de conclusão de curso I}

A Tabela \ref{tab:cronograma1} apresenta o cronograma de desenvolvimento do trabalho conforme a numeração das atividades abaixo:
\begin{enumerate}
  \item Levantamento de dados históricos (documentos e cotações);
  \item Pré-processamento dos dados históricos, como remoção de \textit{stopwords} e \textit{stemming} \cite{Weiss:textMining,orengo_stemming};
  \item Estudo sobre as principais técnicas de mineração de textos;
  \item Levantamento sobre abordagens de mineração de textos voltadas ao mercado de ações;
  \item Testes preliminares com bibliotecas de mineração de dados;
  \item Redação do artigo;
  \item Submissão do artigo para a banca;
  \item Defesa do trabalho.
\end{enumerate}

\begin{table}[h]
  \begin{center}
    \caption{Cronograma de atividades para o trabalho de conclusão I. \label{tab:cronograma1}}
    \begin{tabular}{|c|c|c|c|c|c|c|}
      \hline
      \bf Atividade & \bf 7/12 & \bf 8/12 & \bf 9/12 & \bf 10/12 & \bf 11/12 & \bf 12/12  \\  \hline \hline
      1 & x & x &   &   &   &   \\ \hline
      2 &   & x & x & x &   &   \\ \hline
      3 & x & x & x &   &   &   \\ \hline
      4 & x & x & x & x &   &   \\ \hline
      5 &   &   &   & x & x &   \\ \hline
      6 &   &   & x & x & x &   \\ \hline
      7 &   &   &   &   & x &   \\ \hline
      8 &   &   &   &   &   & x \\ \hline
    \end{tabular}
  \end{center}
\end{table}

\subsection{Trabalho de conclusão de curso II}

A Tabela \ref{tab:cronograma2} apresenta o cronograma de desenvolvimento do trabalho conforme a numeração das atividades abaixo:
\begin{enumerate}
  \item Definição do \textit{setup} de experimentos e métricas de validação com base nos trabalhos relacionados;
  \item Implementação dos experimentos;
  \item Análise de resultados preliminares e ajuste de melhores configurações de parâmetros para os algoritmos de mineração de texto empregados;
  \item Análise comparativa dos resultados com relação à trabalhos relacionados;
  \item Redação do artigo;
  \item Seminário de andamento;
  \item Submissão do artigo para a banca;
  \item Defesa do trabalho.
\end{enumerate}

\begin{table}[h]
  \begin{center}
    \caption{Cronograma de atividades para o trabalho de conclusão II. \label{tab:cronograma2}}
    \begin{tabular}{|c|c|c|c|c|c|c|}
      \hline
      \bf Atividade & \bf 2/13 & \bf 3/13 & \bf 4/13 & \bf 5/13 & \bf 6/13 & \bf 7/13  \\  \hline \hline
      1 & x &   &   &   &   &   \\ \hline
      2 & x & x & x & x &   &   \\ \hline
      3 &   & x & x & x &   &   \\ \hline
      4 &   &   & x & x &   &   \\ \hline
      5 &   &   & x & x & x &   \\ \hline
      6 &   &   &   & x  &   &   \\ \hline
      7 &   &   &   &   & x &   \\ \hline
      8 &   &   &   &   &   & x \\ \hline
    \end{tabular}
  \end{center}
\end{table}

\renewcommand\refname{Referências}
{\small
  \bibliographystyle{ieee}
  \bibliography{referencias}
}

\end{document}
